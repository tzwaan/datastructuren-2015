%----------------------------------------------------------------------------------------
%	LATEX TECHNISCH RAPPORT TEMPLATE
%	Versie 1.1 (4 februari 2015)
%	Opmerkingen of feedback naar Robert van Wijk
%					(robertvanwijk@uva.nl)
%----------------------------------------------------------------------------------------

%----------------------------------------------------------------------------------------
%	PACKAGES EN DOCUMENT CONFIGURATIE
%----------------------------------------------------------------------------------------

\documentclass[a4paper,12pt]{article}
\usepackage{graphicx}
\usepackage[dutch]{babel}
\usepackage{fancyhdr}
\usepackage{lastpage}
\usepackage{xifthen}
\usepackage{algorithm2e}
\usepackage{hyperref}

%----------------------------------------------------------------------------------------
%	HEADER & FOOTER
%----------------------------------------------------------------------------------------
\pagestyle{fancy}
  \lhead{\includegraphics[width=7cm]{logoUvA}}		%Zorg dat het logo in dezelfde map staat
  \rhead{\footnotesize \textsc {Technisch rapport\\ \opdracht}}
  \lfoot
    {
	\footnotesize \studentA
	\ifthenelse{\isundefined{\studentB}}{}{\\ \studentB}
	\ifthenelse{\isundefined{\studentC}}{}{\\ \studentC}
	\ifthenelse{\isundefined{\studentD}}{}{\\ \studentD}
	\ifthenelse{\isundefined{\studentE}}{}{\\ \studentE}
    }
  \cfoot{}
  \rfoot{\small \textsc {Pagina \thepage\ van \pageref{LastPage}}}
  \renewcommand{\footrulewidth}{0.5pt}

\fancypagestyle{firststyle}
 {
  \fancyhf{}
   \renewcommand{\headrulewidth}{0pt}
   \chead{\includegraphics[width=7cm]{logoUvA}}
   \rfoot{\small \textsc {Pagina \thepage\ van \pageref{LastPage}}}
 }

\setlength{\topmargin}{-0.3in}
\setlength{\textheight}{630pt}
\setlength{\headsep}{40pt}

%----------------------------------------------------------------------------------------
%	DOCUMENT INFORMATIE
%----------------------------------------------------------------------------------------
%Geef bij ieder command het juiste argument voor deze opdracht. Vul het hier in en het komt op meerdere plekken in het document correct te staan.

\newcommand{\titel}{Auto Complete}			%Zelfbedachte titel
\newcommand{\opdracht}{Technisch rapport}			%Naam van opdracht die je van docent gehad hebt
\newcommand{\docent}{Floris Kroon}		%Liefst de naam van diegene die het beoordeeld
\newcommand{\cursus}{Datastructuren}
\newcommand{\vakcode}{5062DATA6Y}			%Te vinden op oa Datanose
\newcommand{\datum}{\today}					%Pas aan als je niet de datum van vanaag wilt hebben
\newcommand{\studentA}{Anonymous}
\newcommand{\uvanetidA}{10208917}
%\newcommand{\studentB}{Naam student 2}			%Uncomment de regel als je met twee studenten werkt
\newcommand{\uvanetidB}{UvAnetID student 2}
%\newcommand{\studentC}{Naam student 3}		%Uncomment de regel als je met drie studenten werkt
\newcommand{\uvanetidC}{UvAnetID student 3}
%\newcommand{\studentD}{Naam student 4}		%Uncomment de regel als je met vier studenten werkt
\newcommand{\uvanetidD}{UvAnetID student 4}
%\newcommand{\studentE}{Naam student 5}			%Uncomment de regel als je met vijf studenten werkt
\newcommand{\uvanetidE}{UvAnetID student 5}

%----------------------------------------------------------------------------------------
%	AUTOMATISCHE TITEL
%----------------------------------------------------------------------------------------
\begin{document}
\thispagestyle{firststyle}
\begin{center}
	\textsc{\Large \opdracht}\\[0.2cm]
		\rule{\linewidth}{0.5pt} \\[0.4cm]
			{ \huge \bfseries \titel}
		\rule{\linewidth}{0.5pt} \\[0.2cm]
	{\large \datum  \\[0.4cm]}

	\begin{minipage}{0.4\textwidth}
		\begin{flushleft}
			\emph{Student:}\\
			{\studentA \\ {\small \uvanetidA \\[0.2cm]}}
				\ifthenelse{\isundefined{\studentB}}{}{\studentB \\ {\small \uvanetidB \\[0.2cm]}}
				\ifthenelse{\isundefined{\studentC}}{}{\studentC \\ {\small \uvanetidC \\[0.2cm]}}
				\ifthenelse{\isundefined{\studentD}}{}{\studentD \\ {\small \uvanetidD \\[0.2cm]}}
				\ifthenelse{\isundefined{\studentE}}{}{\studentE \\ {\small \uvanetidE \\[0.2cm]}}
		\end{flushleft}
	\end{minipage}
~
	\begin{minipage}{0.4\textwidth}
		\begin{flushright}
			\emph{Docent:} \\
			\docent \\[0.2cm]
			\emph{Cursus:} \\
			\cursus \\[0.2cm]
			\emph{Vakcode:} \\
			\vakcode \\[0.2cm]
		\end{flushright}
	\end{minipage}\\[1 cm]
\end{center}

%----------------------------------------------------------------------------------------
%	INHOUDSOPGAVE EN ABSTRACT
%----------------------------------------------------------------------------------------
%Niet doen bij korte rapporten

%\tableofcontents
%\begin{abstract}
%\lorem[13]
%\end{abstract}

%----------------------------------------------------------------------------------------
%	INTRODUCTIE
%----------------------------------------------------------------------------------------

\section{Introductie}
Het doel van deze opdracht is om een automatische aanvulfunctie te maken voor
woorden, gebaseerd op een woordenboek dat wordt opgebouwd uit een
gespecificeerde tekst. Het moet daarbij ook mogelijk zijn om woorden uit het
woordenboek te verwijderen. Het de bedoeling dat een trie-datastructuur
gebruikt wordt.

%\subsection{Definities}
%\subsection{Vraagstelling}

%----------------------------------------------------------------------------------------
%	METHODE
%----------------------------------------------------------------------------------------

\section{Methode}
De `trie' datastructuur lijkt op een normale `tree' datastructuur. Echter wordt
er geen data in de nodes opgeslagen, maar in de edges van de trie. Zo zal er
voor alle woorden in het woordenboek beginnend met dezelfde letter slechts
\'e\'en node aangemaakt worden. De edge heeft daarbij de waarde van de letter,
en de node bevat alleen nieuwe edges naar de volgende letters van de
verschillende woorden.

\subsection{Algoritme}
Er is gekozen voor een recursieve aanpak van deze datastructuur, aangezien
iedere node zich hetzelfde gedraagt als zowel zijn children als zijn parents.

Op het moment bevat iedere node een pointer-array naar zijn children, en een
flag of er een woord eindigt op die node. Echter zorgt dit ervoor dat er geen
letters buiten het standaard alfabet gebruikt kunnen worden.

%\subsection{Diagram}
%\subsection{Procedure}
%\subsection{Software en apparatuur}

%----------------------------------------------------------------------------------------
%	RESULTATEN
%----------------------------------------------------------------------------------------

\section{Resultaten}

Er zijn nog geen testresultaten.

%----------------------------------------------------------------------------------------
%	DISCUSSIE
%----------------------------------------------------------------------------------------

\section{Discussie}

Er is nog geen discussie.

%\subsection{Implicaties en aanbevelingen}
%\subsection{Conclusie}


%----------------------------------------------------------------------------------------
%	REFERENTIES
%----------------------------------------------------------------------------------------
%Meer informatie hierover volgt in blok 5 van jaar 1.

\bibliographystyle{acm}

%----------------------------------------------------------------------------------------
%	BIJLAGEN
%----------------------------------------------------------------------------------------

%\section{Bijlage A}
%\section{Bijlage B}
%\section{Bijlage C}

\end{document}
