%----------------------------------------------------------------------------------------
%	LATEX TECHNISCH RAPPORT TEMPLATE
%	Versie 1.1 (4 februari 2015)
%	Opmerkingen of feedback naar Robert van Wijk
%					(robertvanwijk@uva.nl)
%----------------------------------------------------------------------------------------

%----------------------------------------------------------------------------------------
%	PACKAGES EN DOCUMENT CONFIGURATIE
%----------------------------------------------------------------------------------------

\documentclass[a4paper,12pt]{article}
\usepackage{graphicx}
\usepackage[dutch]{babel}
\usepackage{fancyhdr}
\usepackage{lastpage}
\usepackage{xifthen}
\usepackage{algorithm2e}
\usepackage{hyperref}

%----------------------------------------------------------------------------------------
%	HEADER & FOOTER
%----------------------------------------------------------------------------------------
\pagestyle{fancy}
  \lhead{\includegraphics[width=7cm]{logoUvA}}		%Zorg dat het logo in dezelfde map staat
  \rhead{\footnotesize \textsc {Technisch rapport\\ \opdracht}}
  \lfoot
    {
	\footnotesize \studentA
	\ifthenelse{\isundefined{\studentB}}{}{\\ \studentB}
	\ifthenelse{\isundefined{\studentC}}{}{\\ \studentC}
	\ifthenelse{\isundefined{\studentD}}{}{\\ \studentD}
	\ifthenelse{\isundefined{\studentE}}{}{\\ \studentE}
    }
  \cfoot{}
  \rfoot{\small \textsc {Pagina \thepage\ van \pageref{LastPage}}}
  \renewcommand{\footrulewidth}{0.5pt}

\fancypagestyle{firststyle}
 {
  \fancyhf{}
   \renewcommand{\headrulewidth}{0pt}
   \chead{\includegraphics[width=7cm]{logoUvA}}
   \rfoot{\small \textsc {Pagina \thepage\ van \pageref{LastPage}}}
 }

\setlength{\topmargin}{-0.3in}
\setlength{\textheight}{630pt}
\setlength{\headsep}{40pt}

%----------------------------------------------------------------------------------------
%	DOCUMENT INFORMATIE
%----------------------------------------------------------------------------------------
%Geef bij ieder command het juiste argument voor deze opdracht. Vul het hier in en het komt op meerdere plekken in het document correct te staan.

\newcommand{\titel}{Programmeerkeuzes bij Check}			%Zelfbedachte titel
\newcommand{\opdracht}{Opdracht 2}			%Naam van opdracht die je van docent gehad hebt
\newcommand{\docent}{Andy Pimentel\\ Floris Kroon}		%Liefst de naam van diegene die het beoordeeld
\newcommand{\cursus}{Datastructuren}
\newcommand{\vakcode}{5062DATA6Y}			%Te vinden op oa Datanose
\newcommand{\datum}{\today}					%Pas aan als je niet de datum van vanaag wilt hebben
\newcommand{\studentA}{Tijmen Zwaan}
\newcommand{\uvanetidA}{10208917}
%\newcommand{\studentB}{Naam student 2}			%Uncomment de regel als je met twee studenten werkt
\newcommand{\uvanetidB}{UvAnetID student 2}
%\newcommand{\studentC}{Naam student 3}		%Uncomment de regel als je met drie studenten werkt
\newcommand{\uvanetidC}{UvAnetID student 3}
%\newcommand{\studentD}{Naam student 4}		%Uncomment de regel als je met vier studenten werkt
\newcommand{\uvanetidD}{UvAnetID student 4}
%\newcommand{\studentE}{Naam student 5}			%Uncomment de regel als je met vijf studenten werkt
\newcommand{\uvanetidE}{UvAnetID student 5}

%----------------------------------------------------------------------------------------
%	AUTOMATISCHE TITEL
%----------------------------------------------------------------------------------------
\begin{document}
\thispagestyle{firststyle}
\begin{center}
	\textsc{\Large \opdracht}\\[0.2cm]
		\rule{\linewidth}{0.5pt} \\[0.4cm]
			{ \huge \bfseries \titel}
		\rule{\linewidth}{0.5pt} \\[0.2cm]
	{\large \datum  \\[0.4cm]}

	\begin{minipage}{0.4\textwidth}
		\begin{flushleft}
			\emph{Student:}\\
			{\studentA \\ {\small \uvanetidA \\[0.2cm]}}
				\ifthenelse{\isundefined{\studentB}}{}{\studentB \\ {\small \uvanetidB \\[0.2cm]}}
				\ifthenelse{\isundefined{\studentC}}{}{\studentC \\ {\small \uvanetidC \\[0.2cm]}}
				\ifthenelse{\isundefined{\studentD}}{}{\studentD \\ {\small \uvanetidD \\[0.2cm]}}
				\ifthenelse{\isundefined{\studentE}}{}{\studentE \\ {\small \uvanetidE \\[0.2cm]}}
		\end{flushleft}
	\end{minipage}
~
	\begin{minipage}{0.4\textwidth}
		\begin{flushright}
			\emph{Docent:} \\
			\docent \\[0.2cm]
			\emph{Cursus:} \\
			\cursus \\[0.2cm]
			\emph{Vakcode:} \\
			\vakcode \\[0.2cm]
		\end{flushright}
	\end{minipage}\\[1 cm]
\end{center}

%----------------------------------------------------------------------------------------
%	INHOUDSOPGAVE EN ABSTRACT
%----------------------------------------------------------------------------------------
%Niet doen bij korte rapporten

%\tableofcontents
%\begin{abstract}
%\lorem[13]
%\end{abstract}

%----------------------------------------------------------------------------------------
%	INTRODUCTIE
%----------------------------------------------------------------------------------------

\section{Introductie}

De opgave die in dit verslag besproken wordt, is relatief simpel: Lees een
bepaalde opstelling van een schaakbord in, en kijk of \'e\'en van de twee
koningen schaak staat.

\subsection{Definities}

De belangrijkste definities zijn hier de regels van het spel schaak. Daarvoor
zou ik u graag \href{http://www.chessvariants.org/d.chess/chess-dutch.html}
{hierheen} verwijzen.


%\subsection{Vraagstelling}

%----------------------------------------------------------------------------------------
%	METHODE
%----------------------------------------------------------------------------------------

\section{Methode}

Om dit voor elkaar te krijgen moet er voor meerdere stukken worden gekeken of
ze een van de twee koningen schaak zetten.\\

Wat ik me als eerste bedacht was dat het checken van alle stukken apart geen
efficiente manier zou zijn, dus ik heb ervoor gekozen om alle checks vanuit de
koning te doen. Dit geeft uiteindelijk hetzelfde resultaat.

Daarnaast heb ik ervoor gekozen om het bord te representeren met een simpele
2-dimensionale array van characters.

\subsection{Algoritme}

Het algoritme zoekt eerst de witte koning. Waarna vanuit die koning \'e\'en
voor \'e\'en de verschillende stukken worden gezocht in hun bijbehorende
richtingen. Voor de loper wordt bijvoorbeeld schuin gekeken, en voor de toren
juist recht. Als een overeenkomend stuk wordt gevonden, dan staat de koning
schaak, en eindigt het algoritme.\\
Wordt er een ander of geen stuk gevonden, dan gaat het algoritme door naar
de volgende check. Wanneer alle checks gedaan zijn en geen of alleen andere
stukken zijn gevonden, dan gaat het algoritme door met de zwarte koning.

Ik heb ervoor gekozen om voor ieder stuk een bijbehorende functie te schrijven
die speciaal voor dat stuk checkt. Deze worden \'e\'en voor \'e\'en
aangeroepen.

%\subsection{Diagram}
%\subsection{Procedure}
%\subsection{Software en apparatuur}

%----------------------------------------------------------------------------------------
%	RESULTATEN
%----------------------------------------------------------------------------------------

\section{Resultaten}

Het algoritme werkt zoals verwacht. Verder zijn er geen bijzonderheden te
noemen.

%----------------------------------------------------------------------------------------
%	DISCUSSIE
%----------------------------------------------------------------------------------------

\section{Discussie}

Het kwam wat later in mijn hoofd op dat ik nu bepaalde richtingen tweemaal
afga. Zoals schuin om te checken of er een loper of een koningin staat. Het zou
waarschijnlijk sneller zijn geweest als ik die checks tegelijkertijd had
gedaan. Hetzelfde geldt voor de kongingin en de toren.

%\subsection{Implicaties en aanbevelingen}
%\subsection{Conclusie}


%----------------------------------------------------------------------------------------
%	REFERENTIES
%----------------------------------------------------------------------------------------
%Meer informatie hierover volgt in blok 5 van jaar 1.

\bibliographystyle{acm}

%----------------------------------------------------------------------------------------
%	BIJLAGEN
%----------------------------------------------------------------------------------------

%\section{Bijlage A}
%\section{Bijlage B}
%\section{Bijlage C}

\end{document}
